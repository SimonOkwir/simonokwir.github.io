\documentclass[10pt,A4]{article} % Use the custom resume.cls style

\usepackage{verbatim} % usefull to coment out blocs
\usepackage{mycv}

% Define title and author
\title{Curriculum Vitae}
\author{Ant\^onio Horta Ribeiro}


\begin{document}

\maketitle
\small 
\section {Personal Information} 

\begin{itemize}
\item {\bf Birthdate and place:} 1992-10-21, Brazil.
    \item {\bf Work Address:} Room  308, Teknikringen 14, KTH SE-100 44 Stockholm, Sweden
    \item {\bf Emails and phone number:} antonior92@gmail.com; anhr@kth.se,  + 46 70 254 42 85 \item
      {\bf Website:} antonior92.github.io.
\end{itemize}

\section{Academic Positions} % Section title

\begin{itemize}

    \item \shortcventry{ Postdoctoral Associate }
    { Fev. 2022 -   Now  }
    { KTH Royal Institute of Technology, Stockholm, Sweden  }{}

    \item \shortcventry{ Postdoctoral Fellow }
    { Fev. 2021 -   Jan. 2023  }
    { Department of Information Technology, Uppsala University, Uppsala, Sweden  }{}

    \item \shortcventry{ Postdoctoral Associate }
    { Mar. 2020 -   Fev. 2021  }
    { Department of Computer Science, UFMG, Belo Horizonte, Brasil  }{}

\end{itemize}
  
\section{Education} % Section title

\begin{itemize}

    \item \shortcventry{ Ph.D., Electrical Engineering }
    { Aug. 2017 - Mar. 2020 }
    { Universidade Federal de Minas Gerais (UFMG),Brazil }{}

    \item \shortcventry{ M.Sc., Electrical Engineering }
    { Jan. 2016 - Jul. 2017 }
    { Universidade Federal de Minas Gerais (UFMG),Brazil }{}

    \item \shortcventry{ B.S.E., Electrical Engineering }
    { Jan. 2016 - Jul. 2017 }
    { Universidade Federal de Minas Gerais (UFMG),Brazil }{}

\end{itemize}
  

\section{Awards}

\begin{itemize}

    \item \shortcventry{ Benzelius award }
    { 2022 }
    { Royal Society of Sciences in Uppsala }
    { Sweden }
    { I was awarded the Benzelius Award (Benzeliusbelöningarna) due to my 'contributions to fundamental method development in machine learning and control technology, as well as its use to solve important problems in cardiology'. The prize is awarded yearly by the Royal Society of Sciences in Uppsala (Kungliga Vetenskaps-Societeten i Uppsala): the oldest of the royal academies in Sweden, founded in 1710. Named after Erik Benzelius, the prize is awarded to young researchers and comes with the amount of 25000 kronors. }

    \item \shortcventry{ Best Ph.D. Thesis in Engineering and Physical Sciences }
    { 2021 }
    { Universidade Federal de Minas Gerais }
    { Belo Horizonte, Brazil }
    { My Ph.D. thesis was awarded the best Ph.D. thesis defended in 2020 in engineering and physical sciences at the Universidade Federal de Minas Gerais (UFMG), Brazil. In portuguese: Grande Premio de Teses na área de ciências exatas e da terra e engenharias. }

    \item \shortcventry{ Best Ph.D. Thesis in Electrical Engineering }
    { 2021 }
    { Universidade Federal de Minas Gerais }
    { Belo Horizonte, Brazil }
    { My thesis was awarded the best Ph.D. thesis defended in 2020 in the Department of Electrical Engineering at the Universidade Federal de Minas Gerais (UFMG), Brazil. The thesis was then forwarded to compete with the thesis from all other Engineering and Physical Sciences departments at the university (where it was also awarded the best thesis, see the award above). }

    \item \shortcventry{ Young Author Award (Honorable Mention) }
    { 2021 }
    { 19th IFAC Symposium on System Identification }
    { Online }
    { I have been one of the three finalists of the Young Author Award with the paper `Beyond Occam’s Razor in System Identification:  Double-Descent when Modeling Dynamics'. }

    \item \shortcventry{ Best Poster Award }
    { 2019 }
    { SciLifeLab Science Summit }
    { Uppsala, Sweden }
    { I have been awarded the best poster award for the work `Automatic Diagnosis of Short-Duration 12-Lead ECG using a Deep Convolutional Network'. }

    \item \shortcventry{ Travel Award }
    { 2018 }
    { Machine Learning for Health (ML4H) Workshop at NeurIPS }
    { Montreal, Canada }
    { I have been awarded the travel award for the work `Automatic Diagnosis of Short-Duration 12-Lead ECG using a Deep Convolutional Network' and had my expenses covered by the award. }

\end{itemize}
  
\section{Scholarships}

\begin{itemize}

    \item \shortcventry{ UGhent mobility grant }
    { 2023 }
    { Ghent University - 10.000 EUR,Sweden }
    {  }

    \item \shortcventry{ SFVE-A mobility grant }
    { 2023 }
    { French Institute of Sweden,Sweden }
    {  }

    \item \shortcventry{ ELISE mobility grant }
    { 2023 }
    { European Network of AI Excellence Centres,Europe }
    {  }

    \item \shortcventry{ CAPES-PRINT }
    { 2020-2021 }
    { CAPES,Brazil }
    {  }

    \item \shortcventry{ Split-site Ph.D. Scholarship }
    { 2019 }
    { CNPq,Brazil }
    {  }

    \item \shortcventry{ Ph.D. Scholarship }
    { 2018-2020 }
    { CNPq,Brazil }
    {  }

    \item \shortcventry{ M.S. Scholarship }
    { 2016-2017 }
    { CAPES,Brazil }
    {  }

\end{itemize}

\section{Supervision}


  \subsection{\noindent Ph.D.  students, co-supervisor  }
  \begin{itemize}
    
        \item \shortcventry{  Daniel Gedon  }
        { Aug. 2019 - June 2024 (estimated) }
        { Uppsala University, Sweden }
        {  }
     
  \end{itemize}

  \subsection{\noindent M.Sc.  students, supervisor  }
  \begin{itemize}
    
        \item \shortcventry{  Arvid Eriksson  }
        { Jan 2024 - June 2024 (estimated) }
        { KTH, Sweden }
        {  }
     
        \item \shortcventry{  Oscar Larsson  }
        { Feb. 2022 - July 2022 }
        { Uppsala University, Sweden }
        {  }
     
        \item \shortcventry{  Theogene Habineza  }
        { Jan. 2022 - June 2022 }
        { Uppsala University, Sweden }
        {  }
     
  \end{itemize}

  \subsection{\noindent M.Sc.  students, subject reviewer  }
  \begin{itemize}
    
        \item \shortcventry{  Johan Millberg  }
        { Jan. 2023 - June 2023 }
        { Uppsala University, Sweden }
        {  }
     
        \item \shortcventry{  Christie Courtnage  }
        { Jan. 2022 - June 2022 }
        { Uppsala University, Sweden }
        {  }
     
        \item \shortcventry{  Meenal Pathak  }
        { Feb. 2022 - Apr. 2022 }
        { Uppsala University, Sweden }
        {  }
     
  \end{itemize}


\section{Longer scientific visits}

    \begin{itemize}
    
        \item \shortcventry{ Ecolé Normale Superiore / INRIA }
        { March 2023 - June 2023 }
        { Visiting researcher, Paris, France }
        { I was hosted by Francis Bach at the SIERRA team.}
        
        \item \shortcventry{ Uppsala University }
        { September 2018 - September 2019 }
        { Visiting PhD student, Uppsala, Sweden }
        { I was hosted by Thomas B. Schön at the Department of Information Technology.}
        
    \end{itemize}

\section{Teaching}

 \begin{itemize}

    \item \shortcventry{ Advanced Probabilistic Machine Learning  }
    {   Fall - 2022  }
    { Uppsala University, Sweden }
    {  MSc and PhD level, Course responsible and lecturer. Course details: 90 MSc (+11 PhD) students, 5 + 2.5 credits.  \emph{ I was the main responsible for the course. I was involved in lecturing and in preparing the final exam. I also updated the course structure, lecture content and added exercises based on previous year feedback. } }
    
    \item \shortcventry{ Artificial Intelligence and Machine Learning  }
    {   Spring - 2022  }
    { WASP Graduate School, Sweden }
    {  PhD level, Teaching assistant. Course details: 94 students, 6 credits.  \emph{ I was responsible for the design of the course assignment. } }
    
    \item \shortcventry{ Advanced Probabilistic Machine Learning  }
    {   Fall - 2021  }
    { Uppsala University, Sweden }
    {  MSc level, Lecturer. Course details: 125 MSc (+4 PhD) students, 5 + 2.5 credits.  \emph{ I was involved in lecturing and in the preparation of the exam. } }
    
    \item \shortcventry{ The unreasonable effectiveness of overparameterized machine learning models  }
    {   Fall - 2021  }
    { Uppsala University, Sweden }
    {  MSc and PhD level, Course developer and organizer. Course details: 13 students, 3 credits.  \emph{ I was the main responsible for the development of and organization of this new seminar course. I was involved in the choice of papers, in leading the discussion and was responsible for preparing all the assignments. } }
    
    \item \shortcventry{ Deep Learning  }
    {   Spring - 2021  }
    { Uppsala University, Sweden }
    {  PhD level, Teaching assistant. Course details: 54 students, 5 + 3 credits.  \emph{ I was responsible for in preparing the assignment. } }
    
    \item \shortcventry{ Engenharia de Controle (Control Engineering)  }
    {   2nd - 2016  }
    { Universidade Federal de Minas Gerais, Brazil }
    {  BSc level, Teaching assistant. Course details: 50 students, 6 credits.  \emph{ I was responsible for exercise sections and developing the assignment. Also, I was involved in preparing the exam. } }
    
    \item \shortcventry{ Controle Digital  (Digital Control)  }
    {   2nd - 2016  }
    { Universidade Federal de Minas Gerais, Brazil }
    {  BSc level, Teaching assistant. Course details: 40 students, 4 credits.  \emph{ I was responsible for exercise sections and developing the assignment. Also, I was involved in preparing the exam. } }
    
\end{itemize}



\section{Professional activity}

\subsection{Peer reviewing: journal papers}

\begin{itemize}
 \item {\em Nature Communications } (2024),  \item {\em Communications Medicine } (2023),  \item {\em Heart } (2021),  \item {\em IEEE Transactions on Automatic Control } (2021),  \item {\em Heart } (2021),  \item {\em IEEE Transactions on Instrumentation and Measurement } (2021),  \item {\em International Journal of System Science } (2021),  \item {\em Proceedings of the National Academy of Sciences (PNAS) } (2020),  \item {\em Automatica } (2020),  \item {\em IEEE Transactions on Biomedical Engineering } (2020),  \item {\em IEEE Control Systems Letters (L-CSS) } (2020, 2024),  \item {\em Systems and Control Letters } (2020),  \item {\em Chaos, Solutions and Fractals } (2020),  \item {\em Chest } (2020),  \item {\em Journal of Electrocardiology } (2020),  \item {\em Journal of Control, Automation and Electrical Systems } (2015-2018), 
\end{itemize}

\subsection{Peer reviewing: conference papers}

\begin{itemize}
  
  \item {\em International Conference on Artificial Intelligence and Statistics (AISTATS) } (2020-2024),
    
  \item {\em IFAC Symposium on System Identification (SysId) } (2021, 2024),
    
  \item {\em Learning for Dynamics and Control (L4DC) } (2021, 2022),
    
  \item {\em European Control Conference (ECC) } (2020, 2021),
    
  \item {\em IEEE Conference on Decision and Control (CDC) } (2020, 2024),
    
  \item {\em IFAC World Conference } (2017, 2020, 2023),
    
  \item {\em American Control Conference } (2018, 2024),
    
  \item {\em International Conference on Modelling, Identification and Control } (2017),
    
\end{itemize}
  
\subsection{Expert assignments}

\begin{itemize}  
  
  \item ELLIS (European Laboratory for Learning and Intelligent Systems) PhD Program: Recruitment evaluator \hfill {\em 2020 }
  
  \item Co-chair at the session `Parameter Estimation 1' at the 19th IFAC Symposium on System Identification \hfill {\em 2021 }
  
\end{itemize}

\subsection{External examiner in PhD. and M.Sc. defenses}

\begin{itemize}

    \item \cventry{  Najmeh Fayyazifar , Level: Ph.D. }
    { 2022 }
    {  }
    { Edith Cowan University, Australia }
    { {\it Deep learning and neural architecture search for cardiac arrhythmias classification } }

    \item \cventry{  Thiago de Almeida Ushikoshi , Level: M.Sc. }
    { 2022 }
    {  }
    { Universidade Federal de Minas Gerails, Brazil }
    { {\it Learning Nonlinear Dynamics With Echo State Networks } }

\end{itemize}

\subsection{Invited talks}


\begin{itemize}

    
        \item \shortcventry{ Karolinska Institutet @ DDLS fellow public research seminars  }
    { February 2024 }
    { Data-driven ECG analysis  }
    { }{}
    
      
    
        \item \shortcventry{ Royal Institute of Technology, KTH, Sweden @ Division of Robotics, Perception and Learning  }
    { November 2023 }
    { Linear adversarial training, robustness in machine learning and applications to cardiology  }
    { }{}
    
      
    
        \item \shortcventry{ Laboratório Nacional de Computação Científica @ Petrópolis, RJ, Brazil (online)  }
    { October, 2023 }
    { Ataques adversáriais em modelos lineares  }
    { }{}
    
      
    
        \item \shortcventry{ Imperial College, UK @ Imperial Centre for Translational and Experimental Medicine  }
    { July, 2023 }
    { Robustness in large-scale machine learning and its relevance to AI-enabled ECG  }
    { }{}
    
      
    
        \item \shortcventry{ IEEE EMBS @ Germany Chapter, Göttingen (online)  }
    { May, 2023 }
    { The Three Challenges of Using Deep Neural Networks in Electrocardiography  }
    { }{}
    
      
    
        \item \shortcventry{ PUC Rio, Brazil @ Department of Mechanical Engineering  }
    { May, 2023 }
    { Revisitando o princípio da parcimônia na identificação de sistemas e aprendizado de máquina  }
    { }{}
    
      
    
        \item \shortcventry{ INRIA Paris, France @ SIERRA team  }
    { March, 2023 }
    { Overparametrized linear regression under adversarial attacks  }
    { }{}
    
      
    
        \item \shortcventry{ Seminars on Advances in Probabilistic Machine Learning @ Aalto University and ELLIS unit Helsinki  }
    { November 2022 }
    { Adversarial Attacks in Linear Regression  }
    { }{}
    
      
    
        \item \shortcventry{ University of British Columbia, Canada @ Christos Thrampoulidis group (Online)  }
    { June 2022 }
    { Overparameterized Linear Regression under Adversarial Attacks  }
    { }{}
    
      
    
        \item \shortcventry{ University of Luxembourg @ Systems Control Group, LCSB (Online)  }
    { March 2022 }
    { Deep Neural Networks for Automatic ECG Analysis  }
    { }{}
    
      
    
        \item \shortcventry{ International Congress on Electrocardiology (Online)  }
    { April 2021 }
    { Artificial intelligence for ECG classifcation and prediction of the risk of death  }
    { }{}
    
      
    
        \item \shortcventry{ Techinion, Israel @  AIMLab group (Online)  }
    { March 2021 }
    { Artificial intelligence for ECG classification and prediction of the risk of death  }
    { }{}
    
      
\end{itemize}        





\section{Additional work experience} % Section title

\subsection{Open source development}

\begin{itemize}
\item \shortcventry{ \href{https://www.scipy.org}{SciPy} team member }
  {}
  {\em 2017 - 2021}{}
  {}
\end{itemize}

\subsection{Others}
\begin{itemize}
  

    \item \shortcventry{ Software Developer }
    { May. 2017 -   Aug. 2017 }
    { Google Summer of Code }
    { Scipy }
    { I have successfully completed Google Summer of Code program under the mentorship of Matt Haberland, Nikolay Mayorov and Ralf Gommers. My project was the implementation of an interior-point solver for large-scale nonlinear programming problems. The result is the method trust-contr, now openly available as part of the open source scientific library SciPy, in Python. }

    \item \shortcventry{ Hardware Team Intern }
    { Jan. 2015 -   Dec. 2015 }
    { Invent Vision }
    { Belo Horizonte, Brazil }
    { I was part of the hardware development team and worked designing FPGA-based cameras. The major project I have worked on while there was the design and implementation of a stereo camera. }

    \item \shortcventry{ Undergraduate Researcher }
    { Jun. 2013 -   Jan. 2015 }
    { Research and development project with Petrobras Oil Company, UFMG }
    { Belo Horizonte, Brazil }
    { I worked on the development of methods for identification of oil well mathematical models under the supervision of Professor Luis Antonio Aguirre. My position was funded by the Petrobras Oil Company through the Christiano Ottoni Foundation (FCO) in the modality bolsa de iniciação científica. }

\end{itemize} 




\end{document}

