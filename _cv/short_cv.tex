\documentclass[10pt,letterpaper]{article} % Use the custom resume.cls style

\usepackage{verbatim} % usefull to coment out blocs
\usepackage{mycv}

\usepackage{mypublicationlist}

% Define title and author
\title{Short CV}
\author{Ant\^onio Horta Ribeiro}


\pagenumbering{gobble}
\begin{document}

{\Large \textbf{Short Curriculum Vitae} \hfill {\sc  Ant\^onio Horta Ribeiro}}

    \vspace{10pt}
\begin{multicols}{2}
    \small
    \textbf{Current Position:}\\
    \, Postdoctoral Fellow, Uppsala University \\
    \, Department of Information Technology,\\
    {\bf Email:} antonio.horta.ribeiro@it.uu.se\\
    {\bf Website:} antonior92.github.io
\end{multicols}

\subsubsection*{Academic Positions} % Section title


    \shortcventry{ Postdoctoral Fellow }
    { Fev. 2021 -   Now }
    { Department of Information Technology, Uppsala University }
    { Uppsala, Sweden }
    { I am working under the supervision of Thomas Schön on the intersection of machine learning, signal processing, and control theory. }

    \shortcventry{ Postdoctoral Associate }
    { Mar. 2020 -   Fev. 2021 }
    { Department of Computer Science, UFMG }
    { Belo Horizonte, Brasil }
    { I worked on developing new machine learning algorithms and studying its application to engineering and health care. My position was funded by the Brasilian Agency CAPES, through the institutional internalization program (PRINT). }



\subsubsection*{Education} % Section title

    \shortcventry{ Ph.D., Electrical Engineering }
    { Aug. 2017 - Mar. 2020 }
    { Universidade Federal de Minas Gerais (UFMG) }
    { Brazil }
    { I was supervised by Luis Antonio Aguirre and co-supervised by Thomas B. Schon.  I stayed one year, from Sept. 2018 to Sept. 2019, as a guest doctoral student at Uppsala University (Sweden). My Thesis won the award of Best thesis in the Electrical Engineering department and also the best thesis in Engineering and Physical Sciences in the University. }

    \shortcventry{ M.Sc., Electrical Engineering }
    { Jan. 2016 - Jul. 2017 }
    { Universidade Federal de Minas Gerais (UFMG) }
    { Brazil }
    { I was supervised by Luis Antonio Aguirre. I completed 25 credits the equivalent 375 hours in class and my grade pointed average was 5.0 out of 5.0. }

    \shortcventry{ B.S.E., Electrical Engineering }
    { Jan. 2016 - Jul. 2017 }
    { Universidade Federal de Minas Gerais (UFMG) }
    { Brazil }
    { I completed a total of 240  credits (3600 class-hours). And obtained a grade pointed average 4.91 out of 5.00. That is the weighted average of my letter grade (A = 5; B = 4; C = 3; D = 2; E = 1; F = 0) according to the course number of credits. }


\subsubsection*{Awards}


    \shortcventry{ Benzelius award }
    { 2022 }
    { Royal Society of Sciences in Uppsala }
    { Sweden }
    { I was awarded the Benzelius Award (Benzeliusbelöningarna) due to my 'contributions to fundamental method development in machine learning and control technology, as well as its use to solve important problems in cardiology'. The prize is awarded yearly by the Royal Society of Sciences in Uppsala (Kungliga Vetenskaps-Societeten i Uppsala): the oldest of the royal academies in Sweden, founded in 1710. Named after Erik Benzelius, the prize is awarded to young researchers and comes with the amount of 25000 kronors. }
    \shortcventry{ Best Ph.D. Thesis in Engineering and Physical Sciences }
    { 2021 }
    { Universidade Federal de Minas Gerais }
    { Belo Horizonte, Brazil }
    { My Ph.D. thesis was awarded the best Ph.D. thesis defended in 2020 in engineering and physical sciences at the Universidade Federal de Minas Gerais (UFMG), Brazil. In portuguese: Grande Premio de Teses na área de ciências exatas e da terra e engenharias. }
    \shortcventry{ Best Ph.D. Thesis in Electrical Engineering }
    { 2021 }
    { Universidade Federal de Minas Gerais }
    { Belo Horizonte, Brazil }
    { My thesis was awarded the best Ph.D. thesis defended in 2020 in the Department of Electrical Engineering at the Universidade Federal de Minas Gerais (UFMG), Brazil. The thesis was then forwarded to compete with the thesis from all other Engineering and Physical Sciences departments at the university (where it was also awarded the best thesis, see the award above). }
    \shortcventry{ Young Author Award (Honorable Mention) }
    { 2021 }
    { 19th IFAC Symposium on System Identification }
    { Online }
    { I have been one of the three finalists of the Young Author Award with the paper `Beyond Occam’s Razor in System Identification:  Double-Descent when Modeling Dynamics'. }
    \shortcventry{ Best Poster Award }
    { 2019 }
    { SciLifeLab Science Summit }
    { Uppsala, Sweden }
    { I have been awarded the best poster award for the work `Automatic Diagnosis of Short-Duration 12-Lead ECG using a Deep Convolutional Network'. }
    \shortcventry{ Travel Award }
    { 2018 }
    { Machine Learning for Health (ML4H) Workshop at NeurIPS }
    { Montreal, Canada }
    { I have been awarded the travel award for the work `Automatic Diagnosis of Short-Duration 12-Lead ECG using a Deep Convolutional Network' and had my expenses covered by the award. }

\subsubsection*{Scholarships}


    \shortcventry{ CAPES-PRINT }
    { 2020-2021 }
    { CAPES }
    { Brazil }
    { I have been granted a scholarship from the Brasilian Agency CAPES for internacionalization. } 
    \shortcventry{ Split-site Ph.D. Scholarship }
    { 2019 }
    { CNPq }
    { Brazil }
    { I have been granted a scholarship from the Brasilian Agency CNPq for staying one year of my Ph.D. in Uppsala University, Sweden. } 
    \shortcventry{ Ph.D. Scholarship }
    { 2018-2020 }
    { CNPq }
    { Brazil }
    { I have been granted a scholarship from the Brasilian Agency CNPq during my doctoral studies. } 
    \shortcventry{ M.S. Scholarship }
    { 2016-2017 }
    { CAPES }
    { Brazil }
    { I have been granted a scholarship from the Brasilian Agency CAPES during my master studies. } 

\subsubsection*{Supervision}


        \shortcventry{ Daniel Gedon  }
        { Aug. 2019 - Aug. 2024 (estimated) }
        {  Ph.D., co-supervisor  }
        { Uppsala University, Sweden }
        { { Disentangled Representation Learning in Self-Supervised Models } }
        \shortcventry{ Oscar Larsson  }
        { Feb. 2022 - July 2022 }
        {  M.Sc., supervisor  }
        { Uppsala University, Sweden }
        { { Generation and Detection of Adversarial Attacks in the Power Grid } }
        \shortcventry{ Theogene Habineza  }
        { Jan. 2022 - June 2022 }
        {  M.Sc., supervisor  }
        { Uppsala University, Sweden }
        { { Deep Learning-Based Risk Prediction of Atrial Fibrillation Using the 12-lead ECG } }
        \shortcventry{ Christie Courtnage  }
        { Jan. 2022 - June 2022 }
        {  M.Sc., subject reviewer  }
        { Uppsala University, Sweden }
        { { An extension to Semi-Supervised Learning using Shapley Value Data Valuation } }
        \shortcventry{ Meenal Pathak  }
        { Feb. 2022 - Apr. 2022 }
        {  M.Sc., subject reviewer  }
        { Uppsala University, Sweden }
        { { Automated Accounting using Machine Learning } }


\subsubsection*{Teaching}


    \shortcventry{ Advanced Probabilistic Machine Learning  }
    { Fall - 2022  }
    { Course Responsible  - MSc level, 125 students }
    { Uppsala University, Sweden  }
    {}
    \shortcventry{ Artificial Intelligence and Machine Learning  }
    { Spring - 2022  }
    { Teaching assistent  - PhD level, 94 students }
    { WASP Graduate School, Sweden  }
    {}
    \shortcventry{ Advanced Probabilistic Machine Learning  }
    { Fall - 2021  }
    { Lecturer  - MSc level, 125 students }
    { Uppsala University, Sweden  }
    {}
    \shortcventry{ The unreasonable effectiveness of overparameterized machine learning models  }
    { Fall - 2021  }
    { Course organizer  - PhD level, 13 students }
    { Uppsala University, Sweden  }
    {}
    \shortcventry{ Deep Learning  }
    { Spring - 2021  }
    { Teaching assistant  - PhD level, 54 students }
    { Uppsala University, Sweden  }
    {}
    \shortcventry{ Engenharia de Controle (Control Engineering)  }
    { 2nd - 2016  }
    { Teaching assistant  - BSc level, 50 students }
    { Universidade Federal de Minas Gerais, Brazil  }
    {}
    \shortcventry{ Controle Digital  (Digital Control)  }
    { 2nd - 2016  }
    { Teaching assistant  - BSc level, 40 students }
    { Universidade Federal de Minas Gerais, Brazil  }
    {}



\subsubsection*{Bibliometrics}

    \footnotesize
{\bf ORCID}:  \href{https://orcid.org/0000-0003-3632-8529}{0000-0003-3632-8529};
{\bf DBLP}: \href{https://dblp.org/pid/202/1699.html}{202/1699};
{\bf SCOPUS ID}: \href{https://www.scopus.com/authid/detail.uri?authorId=57191699148}{57191699148} ---
    Citations: 7651, h-index: 7, Documents: 23  (2022-09-12);
{\bf Google Scholar}: \href{https://scholar.google.com.br/citations?user=5t_sZdMAAAAJ}{Antonio H. Ribeiro} ---
    Citations: 13217, h-index: 11, i10-index: 12 (2022-09-15).
%{\bf Lattes CV}: \href{http://lattes.cnpq.br/0898576944135254}{0898576944135254}



\subsubsection*{Selected Publications}
\scriptsize
\begin{refsection}
    \renewcommand*{\bibfont}{\normalfont\small}
    \DeclareFieldFormat{labelnumberwidth}{}
    
            \nocite{ ribeiro_automatic_2020a}
    
            \nocite{ lima_deep_2021}
    
            \nocite{ ribeiro_smoothness_2020}
    
            \nocite{ ribeiro_exploding_2020}
    
            \nocite{ ribeiro_parallel_2018}
    
      \printbibliography[omitnumbers=true,heading=none]
\end{refsection}



\end{document}

\newpage





\begin{center}
{\LARGE \bf Additional information \par}
\end{center}
\vspace{20pt}

\section*{Contact References}

I include here the contact information of two references: Luis Antonio Aguirre was my supervisor during my Ph.D.
Thomas Schön is supervising me through my postdoc and is the leader of the group I am now in.
He also hosted me in Uppsala as visiting student during one year of my Ph.D.


\vspace{12pt}
\begin{center}
\begin{tcolorbox}[width=6in, standard jigsaw, opacityback=0]
    \vspace{-4pt}
    \footnotesize
    {\bf Name}: Thomas B. Sch\"on  \\
    {\bf Instituition}: Uppsala University (Sweden) \\
    {\bf Position}: Professor at the Department of Information Technology.\\
    {\bf Email}: thomas.schon@it.uu.se
\end{tcolorbox}
\begin{tcolorbox}[width=6in, standard jigsaw, opacityback=0]
    \vspace{-4pt}
    \footnotesize
    {\bf Name}: Luis Antonio Aguirre \\
    {\bf Instituition}: Federal University of Minas Gerais (Brazil) \\
    {\bf Position}: Professor at the Electronic Engineering Department.\\
    {\bf Email}: aguirre@ufmg.br; aguirre.zanon@gmail.com
\end{tcolorbox}
\end{center}

\vspace{28pt}



\end{document}